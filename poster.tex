% Unofficial University of Bridgeport Poster Template
% https://github.com/andiac/gemini-cam
% a fork of https://github.com/anishathalye/gemini

\documentclass[final]{beamer}

% ====================
% Packages
% ====================

\usepackage[T1]{fontenc}
\usepackage{lmodern}
% \usepackage[size=custom,width=120,height=72,scale=1.0]{beamerposter}

%Change this section based on what we require
%\usepackage[size=custom,width=91.44,height=60.96,scale=1.0] {beamerposter} % centimeter - Standard Landscape

\usepackage[size=custom,width=60.96,height=91.44,scale=1.0] {beamerposter} % centimeter - UBRise Portrait

\usetheme{gemini}
\usecolortheme{UBridgeport}

\usepackage{graphicx}
\usepackage{booktabs}
\usepackage{tikz}
\usepackage{pgfplots}
\pgfplotsset{compat=1.14}
\usepackage{anyfontsize}

% ====================
% Lengths
% ====================

% If you have N columns, choose \sepwidth and \colwidth such that
% (N+1)*\sepwidth + N*\colwidth = \paperwidth
\newlength{\sepwidth}
\newlength{\colwidth}
\setlength{\sepwidth}{0.025\paperwidth}
\setlength{\colwidth}{0.3\paperwidth}

\newcommand{\separatorcolumn}{\begin{column}{\sepwidth}\end{column}}

% ====================
% Title
% ====================
\section{Header}
\title{LawnBot: Affordable Autonomous Lawnmower}

\title{LawnBot: Affordable Autonomous Lawnmower}
\author{Juan Rosario Medina \inst{1} \and Huy Huong \inst{1} \and Alex Armatis \inst{1} \and Carlos Lara \inst{2}}
\institute[UB]{
    University of Bridgeport \inst{1} \\
    School of Computer Science \& Engineering \inst{2} 
}

% ====================
% Logo (optional)
% ====================
% Refer to https://github.com/k4rtik/uchicago-poster
% logo: https://communications.admin.ox.ac.uk/communications-resources/visual-identity/identity-guidelines/the-oxford-logo
% use this to include logos on the left and/or right side of the header:
\logoright{\includegraphics[height=7cm]{logos/UB_Logo.png}}
\logoleft{\includegraphics[height=3cm]{logos/ASEE.png}}

% ====================
% Body
% ====================

\begin{document}

\begin{frame}[t]
\begin{columns}[t]
\separatorcolumn

\begin{column}{\colwidth}
\section{Abstract}
  \begin{block}{Abstract}

    %This WIP Paper is for a student senior design project. 
    The concept of Autonomous Vehicle have spiked proportionally to the investments of Artificial Intelligence and Machine Learning training with real-world data sets [Potential Article quote here]. A great example of this approach is implementing it to a lawnmower, allowing a time-free schedule to consumers. Today's products are trustworthy, but a burdened due to their high starting investment, unexpected costs of maintenance, and unexpected reaction varying the terrain. We aim to minimize the mentioned cons while maximizing the performance with less exposure of software techniques and more emphasis with its design. The term "autonomous" is performing an action with a small to no exposure of human aid. Many of our project's competitors use this practice optimally, we wanted to replicate these approaches but forcing a tighter budget to see how far we can reach to define our project as "autonomous". Realizing our microcontroller's performance, it could not handle most pathing algorithm let alone being compatible with it. We decided to make our project mimic the user's input to then replicate it when needed. As mentioned before, we give you a product for you to store a data set and "Autonomously" replicates it.


  \end{block}

  \begin{block}{Hardware List}
    \begin{itemize}
      \item \textbf{Outer Shell (PLA)}
      \item \textbf{ESP32-WROOM-32}
      \item \textbf{Sensor(s):} GPS/RTK sensors, Ultrasonic Sensors, Rain Sensors
      \item \textbf{Equipment:} Motor Driver, Voltage Regulator, Battery, Wheels, Fuse, Terminal Block, Mower Blade
    \end{itemize}

    

  \end{block}

  \begin{block}{Objectives}
    \begin{itemize}
      \item \textbf Understand the combination of lawnmower's traditional method with software approaches.
      \item \textbf Understand the equipment's purposes for autonomous usage.
      \item \textbf Software limitations with a cheap Microcontroller.
      \item \textbf Improve design for performance.
      \item \textbf Understand the floor and ceiling approaches that are known as "Autonomous".
      \item \textbf Make it profitable.
    \end{itemize}

  \end{block}

\end{column}

\separatorcolumn

\begin{column}{\colwidth}

  \begin{block}{Methodology}
  For a lawnmower, the consumer wants to know the physical attributes before purchasing the product. Its good to know for how long it can work, its cutting height, its cutting width and how can it be handled on a dynamic environment.
  \begin{heading}{Design:}
  \end{heading}

    Since the very beginning, we took the approach of adding a fan; this will improve cutting without changing the cutting height our our robot. The case was designed and 3D printed using PLA.

  \begin{heading}{Software:}
  \end{heading}

    Our project is embedded with 3 modes:
    \begin{itemize}
        \item \textbf{Manual Mode:}
        The user manually controls the lawnmower via bluetooth with you phone (?), these movements will be saved as 
        %Coordinates, taken from the GPS/RTK, and saved through the connected SD Card
        %Movement, taken from how long the user pressed a specific button for a certain period of time, and saved through the connected SD Card 
        \item \textbf{Automatic Mode:}
        Once Manual Mode is completed, you have to place the robot in its origin point and watch how it mimics the user's input from the data collected. The ultrasonic sensors and Rain sensors will provide aid for any dynamic obstacles detected, manipulating the robot's behavior if the conditions are met. If any of the conditions take place, the robot will iterate the data backwards, going back to the first point.  
        \item \textbf{Idle Mode:}
        When there's no SD Card connected to the robot, you can use it as a RC Car to roam around the environment. This is good for setting up the robot properly before going to Manual Mode
        
    \end{itemize}

  \end{block}

  \begin{heading}{Data Collection:}

    Something something something something
  \end{heading}
  \begin{block}{Flowchart}
    \begin{figure}
      \centering
      \begin{tikzpicture}
        \begin{axis}[
            scale only axis,
            no markers,
            domain=0:2*pi,
            samples=100,
            axis lines=center,
            axis line style={-},
            ticks=none]
          \addplot[red] {sin(deg(x))};
          \addplot[blue] {cos(deg(x))};
        \end{axis}
      \end{tikzpicture}
      \caption{Another figure caption.}
    \end{figure}
  \end{block}
  

  \begin{block}{Nam cursus consequat egestas}

    Nulla eget sem quam. Ut aliquam volutpat nisi vestibulum convallis. Nunc a
    lectus et eros facilisis hendrerit eu non urna. Interdum et malesuada fames
    ac ante \textit{ipsum primis} in faucibus. Etiam sit amet velit eget sem
    euismod tristique. Praesent enim erat, porta vel mattis sed, pharetra sed
    ipsum. Morbi commodo condimentum massa, \textit{tempus venenatis} massa
    hendrerit quis. Maecenas sed porta est. Praesent mollis interdum lectus,
    sit amet sollicitudin risus tincidunt non.

  \end{block}

\end{column}

\separatorcolumn

\begin{column}{\colwidth}

  \begin{exampleblock}{A highlighted block containing some math}

    A different kind of highlighted block.

    $$
    \int_{-\infty}^{\infty} e^{-x^2}\,dx = \sqrt{\pi}
    $$

    Interdum et malesuada fames $\{1, 4, 9, \ldots\}$ ac ante ipsum primis in
    faucibus. Cras eleifend dolor eu nulla suscipit suscipit. Sed lobortis non
    felis id vulputate.

    \heading{A heading inside a block}

    Praesent consectetur mi $x^2 + y^2$ metus, nec vestibulum justo viverra
    nec. Proin eget nulla pretium, egestas magna aliquam, mollis neque. Vivamus
    dictum $\mathbf{u}^\intercal\mathbf{v}$ sagittis odio, vel porta erat
    congue sed. Maecenas ut dolor quis arcu auctor porttitor.

    \heading{Another heading inside a block}

    Sed augue erat, scelerisque a purus ultricies, placerat porttitor neque.
    Donec $P(y \mid x)$ fermentum consectetur $\nabla_x P(y \mid x)$ sapien
    sagittis egestas. Duis eget leo euismod nunc viverra imperdiet nec id
    justo.

  \end{exampleblock}

  \begin{block}{Nullam vel erat at velit convallis laoreet}

    Class aptent taciti sociosqu ad litora torquent per conubia nostra, per
    inceptos himenaeos. Phasellus libero enim, gravida sed erat sit amet,
    scelerisque congue diam. Fusce dapibus dui ut augue pulvinar iaculis.

    \begin{table}
      \centering
      \begin{tabular}{l r r c}
        \toprule
        \textbf{First column} & \textbf{Second column} & \textbf{Third column} & \textbf{Fourth} \\
        \midrule
        Foo & 13.37 & 384,394 & $\alpha$ \\
        Bar & 2.17 & 1,392 & $\beta$ \\
        Baz & 3.14 & 83,742 & $\delta$ \\
        Qux & 7.59 & 974 & $\gamma$ \\
        \bottomrule
      \end{tabular}
      \caption{A table caption.}
    \end{table}


  \end{block}

  \begin{block}{References}

    \nocite{*}
    \footnotesize{\bibliographystyle{plain}\bibliography{poster}}

  \end{block}

\end{column}

\separatorcolumn
\end{columns}
\end{frame}

\end{document}
